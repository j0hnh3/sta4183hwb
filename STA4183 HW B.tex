\documentclass[12pt]{article}

\title{STA4183 Homework B}
\author{John He}
\date{10/02/2022}

\begin{document}

\maketitle

\section{}
Use AV of fund for purchase\\
Fund: annuity-immediate with payments of 1000 for n years\\
Purchase: annuity-immediate with payments of 2061.34 for 10 years\\
Annual effective interest rate is 5\%\\
$n=?$\\
\\
AV of the fund at t = n\\
$1000s_{\overline{n|}0.05}$\\
$1000\frac{1.05^{n}-1}{0.05}$\\
$20000(1.05^{n}-1)$\\
\\
PV of the purchase\\
$2061.34a_{\overline{10|}0.05}$\\
$2061.34\frac{1-1.05^{-10}}{0.05}$\\
$\approx15917.12108$\\
\\
Set the AV of the fund at t = n equal to the PV of the purchase\\
$20000(1.05^{n}-1)\approx15917.12108$\\
$1.05^{n}-1\approx0.795856054$\\
$1.05^{n}\approx1.795856054$\\
$n\ln(1.05)\approx\ln(1.795856054)$\\
$n\approx\frac{\ln(1.795856054)}{\ln(1.05)}$\\
$n\approx12$\\
\\
Answer: (e) 12

\section{}
Ben: annuity-immediate with payments of 1700 for 5 years\\
Jerry: annuity-immediate with payments of 1000 for 10 years\\
When the annual effective interest rate is $i\%$,\\
the PVs of the two annuities are equal\\
$i=?$\\
\\
Ben's PV\\
$1700a_{\overline{5|}\frac{i}{100}}$\\
$1700\frac{1-(1+\frac{i}{100})^{-5}}{\frac{i}{100}}$\\
\\
Jerry's PV\\
$1000a_{\overline{10|}\frac{i}{100}}$\\
$1000\frac{1-(1+\frac{i}{100})^{-10}}{\frac{i}{100}}$\\
\\
Set the PVs equal to each other\\
$1700\frac{1-(1+\frac{i}{100})^{-5}}{\frac{i}{100}}=1000\frac{1-(1+\frac{i}{100})^{-10}}{\frac{i}{100}}$\\
$1.7(1-(1+\frac{i}{100})^{-5})=1-(1+\frac{i}{100})^{-10}$\\
$1.7-1.7(1+\frac{i}{100})^{-5}=1-(1+\frac{i}{100})^{-10}$\\
$(1+\frac{i}{100})^{-10}-1.7(1+\frac{i}{100})^{-5}+0.7=0$\\
\\
$(1+\frac{i}{100})^{-5}=\frac{-(-1.7)\pm\sqrt{(-1.7)^{2}-4(1)(0.7)}}{2(1)}$\\
$(1+\frac{i}{100})^{-5}=\frac{1.7\pm0.3}{2}$\\
\\
$(1+\frac{i}{100})^{-5}=1$\\
$1+\frac{i}{100}=1$\\
$i=0$\\
\\
$(1+\frac{i}{100})^{-5}=0.7$\\
$1+\frac{i}{100}=1.073940924$\\
$\frac{i}{100}=0.073940924$\\
$i\approx7.39$\\
\\
Answer: (b) 7.39

\section{}
Thelma: annuity-immediate with payments of 10 for 10 years\\
Louise: annuity-due with payments of X for 12 years\\
When the annual effective interest rate is $8\%$,\\
the PVs of the two annuities are equal\\
$X=?$\\
\\
Thelma's PV\\
$10a_{\overline{10|}0.08}$\\
$10\frac{1-(1.08)^{-10}}{0.08}$\\
$\approx67.10081399$\\
\\
Louise's PV\\
$X\ddot a_{\overline{12|}0.08}$\\
$X\frac{1-(1.08)^{-12}}{\frac{0.08}{1.08}}$\\
$\approx8.138964258X$\\
\\
Set the PVs equal to each other\\
$67.10081399\approx8.138964258X$\\
$X\approx8.24$\\
\\
Answer: (c) 8.24

\section{}
Perpetuity-immediate with payments of X per year\\
Larry: first n payments\\
Curly: next 2n payments\\
Moe: remaining payments\\
Larry's proportion of the PV of the perpetuity is 30\%\\
Moe's proportion of the PV of the perpetuity is M\\
$M=?$\\
\\
Larry: $0.3Xa_{\overline{\infty|}}=0.3\frac{X}{i}$\\
Moe: $MXa_{\overline{\infty|}}=M\frac{X}{i}$\\
\\
Treat Larry's payments as an annuity-immediate\\
Find PV of Larry's payments\\
$Xa_{\overline{n|}i}=X\frac{1-v^{n}}{i}$\\
\\
Set Larry's proportion of the perpetuity equal to the PV of his payments\\
$0.3\frac{X}{i}=X\frac{1-v^{n}}{i}$\\
$0.3=1-v^{n}$\\
$v^{n}=0.7$\\
\\
Moe's proportion of the perpetuity is equal to the PV of his payments\\
$M\frac{X}{i}=v^{3n}\frac{X}{i}$\\
$M=v^{3n}$\\
$M=(v^{n})^{3}$\\
$M=(0.7)^{3}$\\
$M=0.343$\\
\\
Answer: (d) 0.343

\section{}
Annuity-due with payments of 50 every 3 months for 10 years\\
Annual effective interest rate is $8\%$\\
PV of annuity = ?\\
\\
Find nominal quarterly interest rate\\
$i=0.08=(1+i^{(4)})^{4}-1$\\
$i^{(4)}=(1.08)^{\frac{1}{4}}-1$\\
$i^{(4)}\approx0.019426547$\\
\\
$n=4*10=40$\\
$PV=50[1+(1.019426547)^{-1}+(1.019426547)^{-2}+...+(1.019426547)^{-39}]$\\
$PV=50\frac{1-(1.019426547)^{-40}}{1-(1.019426547)^{-1}}$\\
$PV\approx1408.47$\\
\\
Answer: (e) 1408.47

\section{}
Mario: perpetuity-immediate with payments of 1500 every 4-year period\\
Luigi: perpetuity-immediate with payments of 100 every 4-month period\\
The effective annual interest rate is the same for both perpetuities\\
PV of Mario's perpetuity is 3750\\
PV of Luigi's perpetuity = ?\\
\\
$PV_M=3750=\lim_{n\to\infty}1500[(1+i)^{-4}+(1+i)^{-8}+...+(1+i)^{-n}]$\\
$3750=\lim_{n\to\infty}1500(1+i)^{-4}\frac{1-(1+i)^{-n}}{1-(1+i)^{-4}}$\\
$3750=\lim_{n\to\infty}1500\frac{1-(1+i)^{-n}}{(1+i)^{4}-1}$\\
$3750=\frac{1500}{(1+i)^{4}-1}$\\
$i=(\frac{1500}{3750}+1)^{\frac{1}{4}}-1$\\
$i\approx0.087757306$\\
\\
Find the PV of Luigi's perpetuity\\
\\
Find nominal triannual interest rate\\
$i\approx0.087757306=(1+i^{(3)})^{3}-1$\\
$i^{(3)}\approx(1.087757306)^{\frac{1}{3}}-1$\\
$i^{(3)}\approx0.028436156$\\
\\
$PV_L=\lim_{n\to\infty}100[(1.028436156)^{-1}+(1.028436156)^{-2}+...+(1.028436156)^{-n}]$\\
$PV_L=\lim_{n\to\infty}100(1.028436156)^{-1}\frac{1-(1.028436156)^{-n}}{1-(1.028436156)^{-1}}$\\
$PV_L=\lim_{n\to\infty}100\frac{1-(1.028436156)^{-n}}{1.028436156-1}$\\
$PV_L=\frac{100}{0.028436156}$\\
$PV_L\approx3517$\\
\\
Answer: (b) 3517

\section{}
Gandalf: annuity-immediate with payments of 1000 every quarter\\
Annual nominal interest rate of 5.4\% convertible monthly\\
01/01/2015: Account balance is X\\
08/01/2020: Account balance is 1.7X\\
$X=?$\\
\\
01/01/2015 to 07/01/2020: 22 quarters (3 months per quarter)\\
07/01/2020 to 08/01/2020: 1 month\\
01/01/2015 to 08/01/2020: 67 months\\
\\
The AV of Gandalf's annuity-immediate is 1.7X\\
$i=(1+\frac{0.054}{12})^{12}-1$\\
$i\approx0.055356752$\\
\\
Find nominal quarterly interest rate\\
$i\approx0.055356752=(1+i^{(4)})^{4}-1$\\
$i^{(4)}=(1.055356752)^{\frac{1}{4}}-1$\\
$i^{(4)}\approx0.013560841$\\
\\
Find nominal monthly interest rate\\
$i^{(12)}=\frac{0.054}{12}$\\
$i^{(12)}=0.0045$\\
\\
$AV_G=1.7X=X(1.0045)^{67}+(1000[1+1.013560841^1+1.013560841^2+1.013560841^{21}])1.0045$\\
$1.7X=X(1.0045)^{67}+(1000\frac{1.013560841^{22}-1}{1.013560841-1})1.0045$\\
$1.7X\approx1.350971081X+25549.37987$\\
$0.349028919X\approx25549.37987$\\
$X\approx73201$\\
\\
Answer: (e) 73201

\section{}
Gilligan has two options to pay for the purchase of a new boat:\\
1: pay X\\
2: annuity-due with payments of $\frac{X}{11}$ each month for a year\\
If the PV of both options is equal,\\
annual effective interest rate=?\\
\\
$i=(1+i^{(12)})^{12}-1$\\
$i^{(12)}=(1+i)^{\frac{1}{12}}-1$\\
$i^{(12)}+1=(1+i)^{\frac{1}{12}}$\\
\\
$PV_1=PV_2$\\
$X=\frac{X}{11}[1+(1+i)^{-\frac{1}{12}}+...+(1+i)^{-\frac{11}{12}}]$\\
$X=\frac{X}{11}\frac{1-(1+i)^{-\frac{12}{12}}}{1-(1+i)^{-\frac{1}{12}}}$\\
$1=\frac{1}{11}\frac{1-(1+i)^{-1}}{1-(1+i)^{-\frac{1}{12}}}$\\
$11-11(1+i)^{-\frac{1}{12}}=1-(1+i)^{-1}$\\
$0=11(1+i)^{-\frac{1}{12}}-(1+i)^{-1}-10$\\
$i\approx0.21314$\\
\\
Answer: (e) 21.3\%

\section{}
Lori's college fund for daughter: annuity-due with payments of X every month for 20 years\\
The fund earns $i=0.07$\\
Lori's withdrawals: annuity-due with payments of 35000 annually for 4 years (year 18-21)\\
$AV_{Deposits}=AV_{Withdrawals}$ @ Year 21 (240 months + 1 year)\\
\\
$i=0.07=(1+i^{(12)})^{12}-1$\\
$i^{(12)}=(1.07)^{\frac{1}{12}}-1$\\
$i^{(12)}\approx0.005654145$\\
\\
$AV_{Deposits}$ @ Year 21 (240 months + 1 year)\\
$(X[1.005654145+1.005654145^2+1.005654145^{240}])1.07$\\
$X(1.005654145\frac{1.005654145^{240}-1}{1.005654145-1})(1.07)$\\
$\approx546.1344853X$\\
\\
$AV_{Withdrawals}$ @ Year 21 (240 months + 1 year)\\
$35000[1.07+1.07^2+1.07^3+1.07^4]$\\
$35000(1.07)\frac{1.07^{4}-1}{1.07-1}$\\
$\approx166275.8654$\\
\\
$546.1344853X\approx166275.8654$\\
$X\approx304.46$\\
\\
Answer: (d) 304.46

\section{}
Laverne: annuity-due with payments of 1000 annually for 10 years\\
Shirley: annuity-due with level payments of X monthly for 10 years\\
At i=0.08, $PV_L=PV_S$\\
Laverne reinvests her payments at an annual effective rate of 9\% per year\\
Shirley reinvests her payments at an annual effective rate of 10\% per year\\
Find the difference in the AVs of their reinvested payments at the end of 10 years\\
\\
Present value of Laverne\\
$PV_L=1000[1+1.08^{-1}+1.08^{-2}+...+1.08^{-9}]$\\
$PV_L=1000\frac{1-1.08^{-10}}{1-1.08^{-1}}$\\
$PV_L\approx7246.887911$\\
\\
Present value of Shirley\\
$n=12*10=120$\\
$i=0.08=(1+i^{(12)})^{12}-1$\\
$i^{(12)}=(1.08)^{\frac{1}{12}}-1$\\
$i^{(12)}\approx0.00643403$\\
$PV_S=X[1+1.00643403^{-1}+1.00643403^{-2}+...+1.00643403^{-119}]$\\
$PV_S=X\frac{1-1.00643403^{-120}}{1-1.00643403^{-1}}$\\
$PV_S\approx83.9691969X$\\
\\
$7246.887911\approx83.9691969X$\\
$X\approx86.30412316$\\
\\
Accumulated value of Laverne\\
$i=0.09$\\
$AV_L=1000[1.09^{1}+1.09^{2}+...+1.09^{10}]$\\
$AV_L=1000(1.09)\frac{1.09^{10}-1}{1.09-1}$\\
$AV_L\approx16560.29339$\\
\\
Accumulated value of Shirley\\
$i=0.10=(1+i^{(12)})^{12}-1$\\
$i^{(12)}=(1.1)^{\frac{1}{12}}-1$\\
$i^{(12)}\approx0.00797414$\\
$AV_S=(86.30412316)[1.00797414+1.00797414^{2}+...+1.00797414^{119}]$\\
$AV_S=(86.30412316)(1.00797414)\frac{1.00797414^{120}-1}{1.00797414-1}$\\
$AV_S\approx17386.62145$\\
\\
$AV_S-AV_L$\\
$17386.62145-16560.29339$\\
$\approx826.33$\\
\\
Answer: (b) 826.33

\end{document}